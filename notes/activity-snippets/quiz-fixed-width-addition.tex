%! app: TODOapp
%! outcome: TODOoutcome

Recall the definitions of signed integer representations from class: 
sign-magnitude and 2s complement.

\begin{enumerate}
    \item In binary fixed-width addition (adding one bit at time, using 
    the usual column-by-column and carry arithmetic, and ignoring the carry 
    from the  leftmost column), we get: 
    \begin{align*}
        &1110  \qquad  \text{first summand}\\
        +&0100 \qquad  \text{second summand}\\
        &\overline{0010} \qquad \text{result}
    \end{align*}
    Select all and only the  true  statements below:
    \begin{enumerate}
        \item When interpreting each of the summands and the result in binary fixed-width 4, 
        the result represents the actual value of the sum of the summands.
        \item When interpreting each of the summands and the sum in sign-magnitude width 4, the result  
        represents the actual value of the sum of the summands.
        \item When interpreting each of the summands and the sum in 2s complement width 4, the result 
        represents the actual value of the sum of the summands.
    \end{enumerate}    
    \item In binary fixed-width addition (adding one bit at time, using the 
    usual column-by-column and carry arithmetic, and ignoring the carry from the 
    leftmost column), we get: 
    \begin{align*}
        &0110  \qquad  \text{first summand}\\
        +&0111 \qquad  \text{second summand}\\
        &\overline{1101} \qquad \text{result}
    \end{align*}
    Select all and only the  true  statements below:
    \begin{enumerate}
        \item When interpreting each of the summands and the result in binary fixed-width 4, 
        the result represents the actual value of the sum of the summands.
        \item When interpreting each of the summands and the sum in sign-magnitude width 4, 
        the result  
        represents the actual value of the sum of the summands.
        \item When interpreting each of the summands and the sum in 2s complement width 4, 
        the result 
        represents the actual value of the sum of the summands.
    \end{enumerate}   
\end{enumerate}