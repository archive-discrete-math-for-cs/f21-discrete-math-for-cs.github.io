%! app: TODOapp
%! outcome: TODOoutcome

The {\bf universal quantification} of predicate $P(x)$ over
domain $U$ is the statement ``$P(x)$ for all values of $x$ in the domain $U$''
and is written $\forall x P(x)$ or $\forall x \in U ~P(x)$. 
When the domain is finite, universal quantification over the domain 
is equivalent to iterated {\it conjunction} (ands).

The {\bf existential quantification} of predicate $P(x)$ 
over domain $U$ is the statement ``There exists an element $x$ 
in the domain $U$ such that $P(x)$'' and is written $\exists x P(x)$
for $\exists x \in U ~P(x)$. 
When the domain is finite, existential quantification over the domain 
is equivalent to iterated {\it disjunction} (ors).

An element for which $P(x) = F$ is called a {\bf counterexample} of $\forall x P(x)$.

An element for which $P(x) = T$ is called a {\bf witness} of $\exists x P(x)$.
