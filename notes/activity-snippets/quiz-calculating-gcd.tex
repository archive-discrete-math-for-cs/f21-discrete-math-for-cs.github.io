%! app: TODOapp
%! outcome: TODOoutcome

We will consider two ways for calculating the gcd. In each 
part of the question, you'll calculate $gcd(~(306, 120)~$).

\begin{enumerate}
\item The first approach uses some of the claims we proved in class
to get the following algorithm:
%! app: TODOapp
%! outcome: TODOoutcome

\begin{algorithm}[caption={Euclidean algorithm for calculating greatest common divisor}]
    procedure $\textit{Euclidean}$($a$: a positive integer, $b$: a positive integer)
    $x$ := $a$
    $y$ := $b$
    while $y \neq 0$
      $r$ := $x \textbf{ mod } y$
      $x$ := $y$
      $y$ := $r$
    return $x$ $\{ \textrm{the result of } gcd(~(a,b)~)\} $
\end{algorithm}

Tracing this algorithm, lines 2 and 3 initialize the variables 
\[
    x:= 306 \qquad y:=120
\]
Entering the while loop, the variable $r$ is initialized to
\[
    r:= 66
\]
because $306 = 2 \cdot 120 + 66$ so $306 \textbf{ mod } 120 = 66$.
Calculate and fill in the updated value of $r$ in each subsequent iteration
of the {\bf while} loop, and then give the value of $gcd(~(306, 120)~$).

\item The second approach uses the representation of positive integers
greater than $1$ as products of primes. To calculate $gcd(~(a,b)~)$ we 
find the prime factorizations of each of $a$ and $b$, and then calculate
the number that results from multiplying together terms $p^c$ where $p$ is 
a prime that appears in {\it both} prime factorizations of $a$ and $b$
and $c$ is the {\it minimum} number of times $p$ appears in the two factorizations.

Select the prime factorizations for $306$ and $120$ and express their 
$gcd$ as a product of powers of primes.

Possible factorizations:
\begin{enumerate}
        \item $306 = 2 \cdot 153$, $120 = 2 \cdot 60$
        \item $306 = 1 \cdot 2 \cdot 3 \cdot 3 \cdot 17$, $120 = 1 \cdot 3 \cdot 5 \cdot 8$
        \item $306 = 2 \cdot 3 \cdot 3 \cdot 17$, $120 = 2 \cdot 2 \cdot 2 \cdot 3 \cdot 5$
\end{enumerate}

Possible $gcd$ choices:
\begin{enumerate}
\item $2$
\item $2 \cdot 3$
\item $5 \cdot 17$
\item $2^3 \cdot 3^2$
\item $2^3 \cdot 3^2 \cdot 5 \cdot 8 \cdot 17$
\end{enumerate}
\end{enumerate}