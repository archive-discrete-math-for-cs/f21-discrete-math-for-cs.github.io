%! app: TODOapp
%! outcome: TODOoutcome

{\it Recall the definitions}: The set of RNA strands $S$ is defined (recursively) by:
\[
\begin{array}{ll}
\textrm{Basis Step: } & \A \in S, \C \in S, \U \in S, \G \in S \\
\textrm{Recursive Step: } & \textrm{If } s \in S\textrm{ and }b \in B \textrm{, then }sb \in S
\end{array}
\]
where $sb$ is string concatenation.

The function \textit{rnalen} that computes the length of RNA strands in $S$ is defined recursively by:
\[
\begin{array}{llll}
& & \textit{rnalen} : S & \to \mathbb{Z}^+ \\
\textrm{Basis Step:} & \textrm{If } b \in B\textrm{ then } & \textit{rnalen}(b) & = 1 \\
\textrm{Recursive Step:} & \textrm{If } s \in S\textrm{ and }b \in B\textrm{, then  } & \textit{rnalen}(sb) & = 1 + \textit{rnalen}(s)
\end{array}
\]

The function \textit{basecount} that computes the number of a given base 
$b$ appearing in a RNA strand $s$ is defined recursively by:
\[
\begin{array}{llll}
& & \textit{basecount} : S \times B & \to \mathbb{N} \\
\textrm{Basis Step:} &  \textrm{If } b_1 \in B, b_2 \in B & \textit{basecount}(~(b_1, b_2)~) & =
        \begin{cases}
            1 & \textrm{when } b_1 = b_2 \\
            0 & \textrm{when } b_1 \neq b_2 \\
        \end{cases} \\
\textrm{Recursive Step:} & \textrm{If } s \in S, b_1 \in B, b_2 \in B &\textit{basecount}(~(s b_1, b_2)~) & =
        \begin{cases}
            1 + \textit{basecount}(~(s, b_2)~) & \textrm{when } b_1 = b_2 \\
            \textit{basecount}(~(s, b_2)~) & \textrm{when } b_1 \neq b_2 \\
        \end{cases}
\end{array}
\]