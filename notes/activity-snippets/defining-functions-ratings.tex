%! app: Recommendation Systems
%! outcome: function and relation definitions, data types

Recall our representation of Netflix users' ratings of movies as $n$-tuples, where
$n$ is the number of movies in the database. 
Each component of the $n$-tuple is $-1$ (didn't like the movie), $0$ 
(neutral rating or didn't watch the movie), or $1$ (liked the movie).

Consider the ratings $P_1 = (-1, 0, 1)$, $P_2 = (1, 1, -1)$, $P_3 = (1, 1, 1)$,
$P_4 = (0,-1,1)$


Which of $P_1$, $P_2$, $P_3$ has movie preferences most similar to $P_4$?

One approach to answer this question: use {\bf functions} to define distance between user preferences.

For example, consider the function 
$d_0: \phantom{the Cartesian product of the set of ratings on 3 movies with itself} \to \phantom{\mathbb{R}}$
given by
\[
d_0 (~(~ (x_1, x_2, x_3), (y_1, y_2, y_3) ~) ~) = \sqrt{ (x_1 - y_1)^2 + (x_2 - y_2)^2 + (x_3 -y_3)^2}
\]


\vfill
\vfill


{\it Extra example:} A new movie is released, and $P_1$ and $P_2$ watch it before $P_3$, and give it
ratings; $P_1$ gives \cmark~and $P_2$ gives \xmark.
Should this movie be recommended to $P_3$? Why or why not?

{\it Extra example:} Define a new function that could be used to compare the $4$-tuples of ratings encoding
movie preferences now that there are four movies in the database.

\vfill
\newpage