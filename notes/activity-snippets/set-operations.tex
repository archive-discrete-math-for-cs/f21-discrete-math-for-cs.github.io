%! app: TODOapp
%! outcome: write set definition, TODOoutcome

\fbox{\parbox{\textwidth}{%
To define a set we can use the roster method, set builder notation, a recursive definition, 
and also we can apply a set operation to other sets. \\

{\bf New! Cartesian product of sets} and {\bf set-wise concatenation of sets of strings}\\


{\bf Definition}: Let $X$ and $Y$ be sets.  The {\bf Cartesian product} of $X$ and $Y$, denoted
$X \times Y$, is the set of all ordered pairs $(x,y)$ where $x \in X$ and $y \in Y$
\[
X \times Y = \{ (x,y) \mid x \in X \text{ and } y \in Y \}
\]
{\bf Definition}: Let $X$ and $Y$ be sets of strings over the same alphabet. The {\bf set-wise concatenation} 
of $X$ and $Y$, denoted $X \circ Y$, is the set of all results of string concatenation $xy$ where $x \in X$ 
and $y \in Y$
\[
X \circ Y = \{ xy \mid x \in X \text{ and } y \in Y \}
\]
}}

{\bf Pro-tip}: the meaning of writing one element next to another like $xy$ depends on the data-types of $x$ and 
$y$. When $x$ and $y$ are strings, the convention is that $xy$ is the result of string concatenation. 
When $x$ and $y$ are numbers, the convention is that $xy$ is the result of multiplication. This is 
(one of the many reasons) why is it very important to declare the data-type of variables before we use them.

{\it Fill in the missing entries in the table}:

\begin{center}
\begin{tabular}{cc}
{\bf  Set} & {\bf Example elements in this set}:\\
\hline 
& \\
$B$ &\A \qquad \C \qquad \G \qquad \U \\
& \\
\hline
& \\
\phantom{$B \times B$} & $(\A, \C)$ \qquad $(\U, \U)$\\
& \\
\hline
& \\
$B \times \{-1,0,1\}$ & \\
& \\
\hline
& \\
$\{-1,0,1\} \times B$ & \\
& \\
\hline
& \\
\phantom{$\{-1,0,1\} \times \{-1,0,1\}  \times \{-1,0,1\} $} & \qquad $(0,0,0)$ \\
& \\
\hline
& \\
$ \{\A, \C, \G, \U \} \circ  \{\A, \C, \G, \U \}$& \\
& \\
\hline
& \\
\phantom{$\{G\} \circ \{G\} \circ \{G\}$} & \qquad $\G\G\G\G$ \\
& \\
\hline

\end{tabular}
\end{center}

\vfill