%! app: TODOapp
%! outcome: TODOoutcome

{\it Recall}: We say $a$ is {\bf congruent to} $b$ \textbf{mod} $n$ 
means $(a, b) \in R_{(\textbf{mod } n)}$. 
A common notation is to write this as $a \equiv b (\textbf{mod } n)$.

We can partition the set of integers using equivalence classes of  $R_{(\textbf{mod } 4)}$

\begin{align*}
    [0]_{R_{(\textbf{mod } 4)}} &= \phantom{ \{ x \in \mathbb{Z} \mid x \equiv 0 ((\textbf{mod } 4)) \} 
    = \{ x \in \mathbb{Z} \mid x \textbf{ mod } 4 = 0 \textbf{ mod } 4 = 0 \} = \{ 4c \mid c \in \mathbb{Z}\} }\\
    [1]_{R_{(\textbf{mod } 4)}} &= \phantom{ \{ x \in \mathbb{Z} \mid x \equiv 1 ((\textbf{mod } 4)) \} 
    = \{ x \in \mathbb{Z} \mid x \textbf{ mod } 4 = 1 \textbf{ mod } 4 = 1 \} = \{ 4c+1 \mid c \in \mathbb{Z}\} }\\
    [2]_{R_{(\textbf{mod } 4)}} &= \phantom{ \{ x \in \mathbb{Z} \mid x \equiv 2 ((\textbf{mod } 4)) \} 
    = \{ x \in \mathbb{Z} \mid x \textbf{ mod } 4 = 2 \textbf{ mod } 4 = 0 \} = \{ 4c+2 \mid c \in \mathbb{Z}\} }\\
    [3]_{R_{(\textbf{mod } 4)}} &= \phantom{ \{ x \in \mathbb{Z} \mid x \equiv 3 ((\textbf{mod } 4)) \} 
    = \{ x \in \mathbb{Z} \mid x \textbf{ mod } 4 = 3 \textbf{ mod } 4 = 3 \} = \{ 4c+3 \mid c \in \mathbb{Z}\} }\\
    [4]_{R_{(\textbf{mod } 4)}} &= \phantom{ \{ x \in \mathbb{Z} \mid x \equiv 4 ((\textbf{mod } 4)) \} 
    = \{ x \in \mathbb{Z} \mid x \textbf{ mod } 4 = 4 \textbf{ mod } 4 = 0 \} = \{ 4c \mid c \in \mathbb{Z}\} }\\
    [5]_{R_{(\textbf{mod } 4)}} &= \phantom{ \{ x \in \mathbb{Z} \mid x \equiv 5 ((\textbf{mod } 4)) \} 
    = \{ x \in \mathbb{Z} \mid x \textbf{ mod } 4 = 5 \textbf{ mod } 4 = 1 \} = \{ 4c+1 \mid c \in \mathbb{Z}\} }\\
    [-1]_{R_{(\textbf{mod } 4)}} &= \phantom{ \{ x \in \mathbb{Z} \mid x \equiv -1 ((\textbf{mod } 4)) \} 
    = \{ x \in \mathbb{Z} \mid x \textbf{ mod } 4 = -1 \textbf{ mod } 4 = 3 \} = \{ 4c+3 \mid c \in \mathbb{Z}\} }
\end{align*}
\[
\mathbb{Z} =  [0]_{R_{(\textbf{mod } 4)}}~ \cup ~[1]_{R_{(\textbf{mod } 4)}} ~\cup~[2]_{R_{(\textbf{mod } 4)}}~\cup~
[3]_{R_{(\textbf{mod } 4)}}
\]





