\documentclass[12pt, oneside]{article}

\usepackage[letterpaper, scale=0.8, centering]{geometry}
\usepackage{fancyhdr}
\setlength{\parindent}{0em}
\setlength{\parskip}{1em}

\pagestyle{fancy}
\fancyhf{}
\renewcommand{\headrulewidth}{0pt}
\rfoot{{\footnotesize Copyright Mia Minnes, 2021, Version \today~(\thepage)}}

\author{CSE20F21}

\newcommand{\instructions}{{\bf For all HW assignments:}

Weekly homework may be done individually or in groups of up to 3 students. 
You may switch HW partners for different HW assignments. 
The lowest HW score will not be included in your overall HW average. 
Please ensure your name(s) and PID(s) are clearly visible on the first page of your homework submission.

All submitted homework for this class must be typed. 
Diagrams may be hand-drawn and scanned and included in the typed document. 
You can use a word processing editor if you like (Microsoft Word, Open Office, Notepad, Vim, Google Docs, etc.) 
but you might find it useful to take this opportunity to learn LaTeX. 
LaTeX is a markup language used widely in computer science and mathematics. 
The homework assignments are typed using LaTeX and you can use the source files 
as templates for typesetting your solutions\footnote{To use this template, copy the source file (extension \texttt{.tex}) 
to your working directory or upload to Overleaf.}.


{\bf Integrity reminders}
\begin{itemize}
\item Problems should be solved together, not divided up between the partners. The homework is
designed to give you practice with the main concepts and techniques of the course, 
while getting to know and learn from your classmates.
\item You may not collaborate on homework with anyone other than your group members.
You may ask questions about the homework in office hours (of the instructor, TAs, and/or tutors) and 
on Piazza (as private notes viewable only to the Instructors).  
You \emph{cannot} use any online resources about the course content other than the class material 
from this quarter -- this is primarily to ensure that we all use consistent notation and
definitions we will use this quarter.
\item Do not share written solutions or partial solutions for homework with 
other students in the class who are not in your group. Doing so would dilute their learning 
experience and detract from their success in the class.
\end{itemize}

}
\usepackage{amssymb,amsmath,pifont,amsfonts,comment,enumerate,enumitem}
\usepackage{currfile,xstring,hyperref,tabularx,graphicx,wasysym}
\usepackage[labelformat=empty]{caption}
\usepackage[dvipsnames,table]{xcolor}
\usepackage{multicol,multirow,array,listings,tabularx,lastpage,textcomp,booktabs}

\lstnewenvironment{algorithm}[1][] {   
    \lstset{ mathescape=true,
        frame=tB,
        numbers=left, 
        numberstyle=\tiny,
        basicstyle=\rmfamily\scriptsize, 
        keywordstyle=\color{black}\bfseries,
        keywords={,procedure, div, for, to, input, output, return, datatype, function, in, if, else, foreach, while, begin, end, }
        numbers=left,
        xleftmargin=.04\textwidth,
        #1
    }
}
{}
\lstnewenvironment{java}[1][]
{   
    \lstset{
        language=java,
        mathescape=true,
        frame=tB,
        numbers=left, 
        numberstyle=\tiny,
        basicstyle=\ttfamily\scriptsize, 
        keywordstyle=\color{black}\bfseries,
        keywords={, int, double, for, return, if, else, while, }
        numbers=left,
        xleftmargin=.04\textwidth,
        #1
    }
}
{}

\newcommand\abs[1]{\lvert~#1~\rvert}
\newcommand{\st}{\mid}

\newcommand{\A}[0]{\texttt{A}}
\newcommand{\C}[0]{\texttt{C}}
\newcommand{\G}[0]{\texttt{G}}
\newcommand{\U}[0]{\texttt{U}}

\newcommand{\cmark}{\ding{51}}
\newcommand{\xmark}{\ding{55}}

 
 
\title{HW7 Function and Relations}
\date{Due: Tuesday, November 30, 2021 at 11:00PM on Gradescope}

\begin{document}
\maketitle
\thispagestyle{fancy}

{\bf In this assignment,}

You will practice determining and justifying whether 
statements are true in multiple contexts.

Instructions and academic integrity reminders for all homework assignments in 
CSE20 this quarter are on the class website and on the hw1-definitions-and-notations
assignment.

You will submit this assignment via Gradescope
(\href{https://www.gradescope.com}{https://www.gradescope.com}) 
in the assignment called ``hw7-functions-and-relations''.

{\bf Resources}: To review the topics you are working with 
for this assignment, see the class material from Weeks 8 and 9.
We will post frequently asked questions and our answers to them in a 
pinned Piazza post.


In your proofs and disproofs of statements below, justify each  step
by reference to  a component of the  following proof  strategies
we  have discussed so far, and/or to relevant definitions and calculations.
\begin{itemize}
    \item A counterexample can be used to prove that  $\forall x P(x)$ is {\bf false}.
    \item  A witness can be used to prove that  $\exists x P(x)$ is {\bf true}.
    \item {\bf Proof of universal by exhaustion}: To prove that $\forall x \, P(x)$
is true when $P$ has a finite domain, evaluate the predicate at {\bf each} domain element to confirm that it is always T.
    \item  {\bf Proof by universal generalization}: To prove that $\forall x \, P(x)$
is true, we can take an arbitrary element $e$ from the domain and show that $P(e)$ is true, without making any assumptions 
about $e$ other than that it comes from the domain.
    \item To  prove  that $\exists x P(x)$ is {\bf false}, write the universal statement that is 
    logically equivalent to its negation and then prove it true using universal generalization.
    \item {\bf Strategies for conjunction}: To prove that $p \land q$ is true, have two subgoals: 
    subgoal (1) prove $p$ 
is  true; and, subgoal (2) prove $q$ is true. To prove that $p \land q$ is false, it's enough to prove that $p$ is false.
 To prove that $p \land q$ is false, it's enough to prove that $q$ is false.
    \item {\bf Proof of Conditional by Direct Proof}: To prove that the implication $p \to q$ is true, 
    we can assume $p$ is true and use that assumption to show $q$ is true.
    \item {\bf Proof of Conditional by Contrapositive Proof}: To prove that the implication $p \to q$ is true, 
    we can assume $\neg q$ is true and use that assumption to show $\neg p$ is true.
    \item {\bf Proof of disjuction using equivalent conditional}: To prove that the 
    disjunction $p \lor q$ is true, we can rewrite it equivalently as $\lnot p \to q$ and
    then use direct proof or contrapositive proof.
    \item {\bf Proof by Cases}: To prove $q$ when we know $p_1 \lor p_2$, show that $p_1 \to q$ and $p_2 \to q$.
    \item
    {\bf Proof by Structural Induction}: To prove that $\forall x \in X \, P(x)$ where $X$ is a recursively defined set, prove two cases:
        
        \begin{tabularx}{\textwidth}{l X}
        Basis Step: & Show the statement holds for elements specified in the basis step of the definition. \\
        Recursive Step: & Show that if the statement is true for each of the elements used to construct
    new elements in the recursive step of the definition, the result holds for these new elements.
    \end{tabularx}
    
    \item {\bf Proof by Mathematical Induction}: To prove a universal quantification over the set of  all integers greater than  or  equal to some base integer $b$:
    
    \begin{tabularx}{\textwidth}{l X}
        Basis Step: & Show the statement holds for $b$. \\
        Recursive Step: & Consider an arbitrary integer $n$ greater than or  equal to  $b$, assume
        (as the {\bf induction hypothesis})  that the property holds  for $n$, and use  this and
        other facts to  prove that  the property holds for $n+1$.
    \end{tabularx}
    
    \item {\bf Proof by Strong Induction} To prove that a universal quantification over the set of all integers greater than or equal to some  base integer $b$ holds,  pick a  fixed nonnegative integer  $j$ and then: \hfill 
    
    \begin{tabularx}{\textwidth}{l X}
        Basis Step: & Show the statement holds for $b$, $b+1$, \ldots, $b+j$. \\
        Recursive Step: & Consider an arbitrary integer $n$ greater than or  equal to  $b+j$, assume
        (as the {\bf strong  induction hypothesis})  that the property holds  for {\bf each of} $b$, $b+1$, \ldots, $n$, 	
        and use  this and
        other facts to  prove that  the property holds for $n+1$.
    \end{tabularx}

    \item {\bf Proof by Contradiction} 

    To prove that a statement $p$ is true, pick another statement $r$ and once we show
    that $\neg p  \to (r \wedge  \neg r)$ then  we can conclude that  $p$ is  true.
    
    {\it Informally} The statement we care about can't possibly be false, so it must be true.
\end{itemize}

\newpage
{\bf Assigned questions}

\begin{enumerate}
    \item Consider the set $U = \mathcal{P}(\mathbb{R})$. 
    
    \begin{enumerate}
        \item ({\it Translation graded for fair effort completeness; Counterexample graded 
        for correctness})  Translate the statement to English and then give a counterexample 
        that could be used to disprove the statement. You do not need to justify your answer.  
        However, if you include clear explanations, 
        we may be able to give partial credit for an answer with some imprecision.

        {\it Note}: your counterexample should specify a value for $A$ and a value for $B$.
        \[
            \forall A \in U ~\forall B \in U~(~( A \subseteq B \to \lnot (~|A| \geq |B|~)~)
        \]
        \item ({\it Translation graded for fair effort completeness; Counterexample graded 
        for correctness})  Translate the statement to English and then give a counterexample 
        that could be used to disprove the statement. You do not need to justify your answer.  
        However, if you include clear explanations, 
        we may be able to give partial credit for an answer with some imprecision.

        {\it Note}: your counterexample should specify a value for $X$ and a value for $Y$.
        \[
            \forall X \in U~\forall Y \in U~( ~X \subseteq \mathbb{Z}~\land~ Y \subseteq \mathbb{Z} ~\to~  |X|  = |Y| ~)
        \]
        \item ({\it Translation graded for fair effort completeness; Witness graded 
        for correctness})  Translate the statement to English and then give a witness 
        that could be used to prove the statement. You do not need to justify your answer.  
        However, if you include clear explanations, 
        we may be able to give partial credit for an answer with some imprecision.

        {\it Note}: your witness should specify a value for $X$ and a value for $Y$.
        \[
            \exists X \in U~ \exists Y \in U (~(~\mathbb{Z} \subseteq X~) \land
            (~\mathbb{Z} \subseteq Y~) \land \neg ( |X| = |Y|)~)
        \]
    \end{enumerate}

    \item ({\it Graded for correctness}\footnote{This means your solution will be
    evaluated not only on the correctness of your answers, but on your ability to 
    present your ideas clearly and logically. You should explain how you arrived at 
    your conclusions, using 
    mathematically sound reasoning. Whether you use formal proof techniques or 
    write a more informal argument for why 
    something is true, your answers should always be well-supported. Your goal 
    should be to convince the reader that 
    your results and methods are sound.})
    The diagonalization argument constructs, for each function 
    $f: \mathbb{N} \to \mathcal{P}(\mathbb{N})$, a set $D_f$ defined as
    \[
    D_f = \{ x \in \mathbb{N} ~|~ x \notin f(x) \}
    \]
    \begin{enumerate}
        \item Define a function $g$ such that $D_g$ is a finite nonempty set, 
        or explain why no such function exists.
        \item Define a function $h$ such that $D_h$ is an infinite set that is a proper subset of 
        $\mathbb{N}$, or explain why no such function exists.
        \item Define a function $k$ such that $D_k$ is a proper superset
        of $\mathbb{N}$ (in other words, $\mathbb{N}$ is a proper subset 
        of $D_k$), or explain why no such function exists.\footnote{
        For sets $A$ and $B$, when the relation $A\subseteq B$ holds we say that $A$ 
        is a subset of $B$ and that $B$ is a superset of $A$. Similarly, when the relation 
        $A\subsetneq B$ holds we say that $A$ is a proper subset of $B$ and that $B$ is a 
        proper superset of $A$. 
        }
    \end{enumerate}    
    
    \item ({\it Graded for correctness}) For each part of this question, you do not need to justify your answer.  
    However, if you include clear explanations, 
    we may be able to give partial credit for an answer with some imprecision.
        
    \begin{enumerate}
    \item Recall that 
    in a movie recommendation system, each 
    user's ratings of movies is represented as a $n$-tuple (with the positive integer $n$ 
    being the number of movies in the database), and each component of 
    the $n$-tuple is an element of the collection $\{-1,0,1\}$. Assume there are five movies in the database, 
    so that each user's ratings
    can be represented as a $5$-tuple. We call $Rt_5$ the set of all ratings $5$-tuples.
    Consider the binary relation on the  set of all 
    $5$-tuples where each  component of the $5$-tuple is an element of the collection $\{-1,0,1\}$:
    \[
        G = \{ (u,v) \in Rt_5 \times Rt_5 \mid \text{the number of $0$s in $u$ is the same as the number of $0$s in $v$} \}
    \]
    This is an equivalence relation (you do not need to prove this).

    Recall that the {\bf equivalence class} of an element $x \in X$ for an equivalence relation $\sim$ on the set $X$ 
    is the set $\{s \in X | (x, s) \in \sim \}$. We write this as $[x]_\sim$.
    
    \begin{enumerate}
        \item Find a ratings $5$-tuple $v$ such that $[v]_{G} = \{v \}$.
        \item Find distinct ratings $5$-tuples $u_1, u_2$ ($u_1 \neq u_2$) whose equivalence classes $[u_1]_{G}$ and $[u_2]_{G}$ have the same size.
        \item Find distinct ratings $5$-tuples $w_1, w_2$ ($w_1 \neq w_2$) whose equivalence classes $[w_1]_{G}$ and $[w_2]_{G}$ have different sizes.
    \end{enumerate}

    \item Let $S_{1,2}$ be the set of RNA strands of length $1$ or $2$, formally 
        \[
            S_{1,2} = \{ s \in S \mid ( rnalen(s) = 1) \lor (rnalen(s) = 2)\}
        \]
    Consider the binary relation on $S_{1,2}$ given by
    \begin{align*}
        P = \{ (s, s') \in S_{1,2} \times S_{1,2} \mid &\text{$s$ is a prefix of $s'$, }\\
        &\text{namely either $s = s'$ or there is 
        some base $b$ such that $sb=s'$} \}
    \end{align*}
    This is a partial ordering (you do not need to prove this).

    Draw the Hasse diagram of $P$.

    \item Consider the set $CP$ of compound propositions that use propositional variables from the set $\{p,q\}$.
    We define the logical equivalence binary relation on this set by 
    \[
        LE = \{ (x, y) \in CP \times CP \mid x \equiv y \}
    \]
    This is an equivalence relation (you do not need to prove this).

    \begin{enumerate}
        \item Give two distinct examples of elements in $[~(p \land \lnot p)~]_{LE}$
        \item Give two distinct examples of elements in $[~(p \to q)~]_{LE}$
    \end{enumerate}

    {\it Bonus - not for credit; do not hand in}: 
    Prove that $G$ is an equivalence relation on
    the set of ratings $5$-tuples.
    Prove that $P$ is a partial ordering on $S_{1,2}$.
    Prove that $LE$ is an equivalence relation on 
    the set of compound propostions that use propositional variables from 
    the set $\{p,q\}$.



    \end{enumerate}

    \item Imagine you are playing the role of Alice in the Diffie Hellman key agreement (exchange) protocol.  
    You and Bob have agreed to use the prime $p = 7$ and its primitive root $a = 3$.
    Your secret integer is $k_1 = 3$.
        
        \begin{enumerate}
        \item  ({\it Graded for fair effort completeness}\footnote{This means 
        you will get full credit so long as your submission demonstrates honest 
        effort to answer the question. You will not be penalized for incorrect answers.}) 
        Calculate the number you send to Bob, 
        $a^{k_1} \textrm{\bf ~mod~} p$.  Use the modular exponentiation algorithm
        for the calculation. Include a trace of the algorithm in your solution.

        

\begin{algorithm}[caption={Modular Exponentation}]
    procedure $modular~exponentiation$($b$: integer; 
                 $n = (a_{k-1}a_{k-2} \ldots a_1 a_0)_2$, $m$: positive integers)
    $x$ := $1$
    $power$ := $b$ mod $m$
    for $i$:= $0$ to $k-1$
      if $a_i = 1$ then $x$:= $(x \cdot power)$ mod $m$
      $power$ := $(power \cdot power)$ mod $m$
    return $x$ $\{x~\textrm{equals}~b^n \textbf{ mod } m\} $
\end{algorithm}         
        \item  ({\it Graded for fair effort completeness}) Bob sends you the number $5$. Compute  your shared key, $\left(a^{k_2}\right)^{k_1}  
        \textrm{\bf ~mod~} p$.
        Hint: $a^{k_2} \textrm{\bf ~mod~} p$ is what Bob sent you.  Include all relevant calculations, annotated with explanations, 
        for full credit.
        
        
        \item ({\it Graded for fair effort completeness}) What are some possible values for Bob's secret integer?  What 
        algorithm are you using to compute them?
        \end{enumerate}
    

\end{enumerate}


\end{document}    
